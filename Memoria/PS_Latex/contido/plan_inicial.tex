\chapter{Planificación Inicial}
\label{chap:plan_inicial}

\section{Iteracións}
O proxecto fragmentarase na iteración de presentación do proxecto (\textit{brainstorming}, preparación dos documentos necesarios) dúas iteracións para o \textit{singleplayer} (desarrollo \textit{software} e probas) e outras dúas iguais para o \textit{multiplayer}.

Para cada modo de xogo, o primeiro que se fará é a vista principal do xogo e os fragmentos das diferentes modalidades que ofrece, como pode ser a elección do modo, ou obter o historial de partidas. Despois
comezarase coa creación da vista do xogo en si e os seus elementos e finalmente comprobarase con tests o seu correcto funcionamento, probar cando unha letra non pertence á palabra
a adiviñar, a condición de gañar ou perder a partida, etc.\\
Se excedese tempo ao rematar a devandita iteración e antes da data de entrega do proxecto plantexaríase facer unha última iteración para implementar unha funcionalidade multixogador \textit{online} e as súas probas correspondentes.
%\begin{enumerate}
%    \item \textbf{Modo singleplayer}
%    \item \textbf{Modo multiplayer}
%\end{enumerate}

\section{Responsabilidades}
Por dispoñibilidade temporal, a idea é que todos os integrantes do grupo implementen as funcionalidades xuntos xa que é improbable que se solapen os traballos de dous recursos diferentes. Todos os membros do grupo terán a mesma porcentaxe de responsabilidade aínda que o esperable é que os traballos se vaian asignando dinamicamente e sobre a marcha do proxecto segundo vaian xurdindo., axudarémonos do control de versións \textit{git} para facilitar o traballo.

\section{Fitos e entregables}
Haberá un total de 2
fitos, un para cada unha das funcionalidades a realizar cada unha cun mes de duración. Primeiro farase o modo \textit{singleplayer} e todo o que conleva, facer todas as vistas e fragmentos para que se poida xogar, se hai problemas ou non será a tarefa dos tests. No segundo , actualizarase a vista para que apareza o outro modo de xogo, e a vista será modificada respecto á do \textit{singleplayer} está pensado so para un xogador.%No terceiro fito, haberá que implementar o código que permita xogar ao forcado en dous diferentes dispositivos.


\section{Incidencias e plans de continxencia}
O principal risco que se vai ter en conta en conta debido á súa probable materialización son os contratempos á hora do desenvolvemento \textit{software}. Este risco está presente en case todos os proxectos \textit{software} e aínda máis cando se trata dun proxecto non profesional e con falta de experiencia na metodoloxía.\\
A solución aplicada para isto é tratar de adiantar o proxecto máximo posible coa mentalidade pesimista de que chegará algún contratempo que retrase o proceso.
%offline e se todo vai ben o online.
%No caso de non xurdiren problemas para implementar o modo online, intentarase ofrecer máis modalidades ou ítems de axuda. %nas funcionalidades singleplayer ou offline, 
%como poden ser moedas virtuais que compren pistas ou descubran letras para ter máis probabilidade de adiviñar a palabra.

 \let\cleardoublepage=\clearpage 