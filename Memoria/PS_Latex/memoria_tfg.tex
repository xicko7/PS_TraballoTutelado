%%%%%%%%%%%%%%%%%%%%%%%%%%%%%%%%%%%%%%%%%%%%%%%%%%%%%%%%%%%%%%%%%%%%%%%%%%%%%%%%
% Preámbulo                                                                    %
%%%%%%%%%%%%%%%%%%%%%%%%%%%%%%%%%%%%%%%%%%%%%%%%%%%%%%%%%%%%%%%%%%%%%%%%%%%%%%%%

\documentclass[11pt,a4paper,titlepage,twoside,openright,openbib]{report}

%%% RELACIÓN DE VARIABLES A PERSONALIZAR %%%
\def\lingua{gal}
%\def\lingua{esp} % descomenta esta liña se redactarás a memoria en español
%\def\lingua{eng} % descomenta esta liña se redactarás a memoria en inglés
\def\nomeA{Xico Fernández Lozano (\href{mailto:xico.fernandez@udc.es}{xico.fernandez})}
\def\nomeB{Mario Páez Marcote (\href{mailto:mario.paez@udc.es}{mario.paez})}% substitúe aquí o teu nome
\def\nomeC{Andrés Filipe Oliveira da Silva (\href{mailto:andres.oliveira@udc.es}{andres.oliveira})}             % substitúe aquí o nome de quen dirixe
\def\titulo{XOGO DO FORCADO PARA ANDROID  ForcApp}

\def\icon{\begin{figure}[hp!]
  \centering
  \includegraphics[width=0.15\textwidth]{imaxes/icon.png}
  \label{fig:icon}
\end{figure}}


%\def\mencion{COMPUTACIÓN}
%\def\mencion{ENXEÑARÍA DO SOFTWARE}
\def\mencion{ENXEÑARÍA DE COMPUTADORES}
%\def\mencion{SISTEMAS DE INFORMACIÓN}
%\def\mencion{TECNOLOXÍAS DA INFORMACIÓN}

%\def\renomearcadros{si} % descomenta esta liña se redactas a memoria en español e prefires que
                         % os "cuadros" e o "índice de cuadros" se renomeen
                         % a "tablas" e "índice de tablas" respectivamente

\usepackage{estilo_tfg}
  
% Lista de paquetes potencialmente interesantes (uso baixo demanda)

% \usepackage{alltt}       % proporciona o entorno alltt, semellante a verbatim pero que respecta comandos
% \usepackage{enumitem}    % permite personalizar os entornos de lista
% \usepackage{eurofont}    % proporciona o comando \euro
% \usepackage{float}       % permite máis opcións para controlar obxectos flotantes (táboas, figuras)
% \usepackage{hhline}      % permie personalizar as liñas horizontais en arrays e táboas
% \usepackage{longtable}   % permite construir táboas que ocupan máis dunha páxina
% \usepackage{lscape}      % permite colocar partes do documento en orientación apaisada
% \usepackage{moreverb}    % permite personalizar o entorno verbatim
% \usepackage{multirow}    % permite crear celdas que ocupan varias filas da mesma táboa
% \usepackage{pdfpages}    % permite insertar ficheiros en PDF no documento
% \usepackage{rotating}    % permite diferentes tipos de rotacións para figuras e táboas
% \usepackage{subcaption}  % permite a inclusión de varias subfiguras nunha figura
% \usepackage{tabu}        % permite táboas flexibles
% \usepackage{tabularx}    % permite táboas con columnas de anchura determinada

%%%%%%%%%%%%%%%%%%%%%%%%%%%%%%%%%%%%%%%%%%%%%%%%%%%%%%%%%%%%%%%%%%%%%%%%%%%%%%%%
% Corpo                                                                        %
%%%%%%%%%%%%%%%%%%%%%%%%%%%%%%%%%%%%%%%%%%%%%%%%%%%%%%%%%%%%%%%%%%%%%%%%%%%%%%%%

\begin{document}

 %%%%%%%%%%%%%%%%%%%%%%%%%%%%%%%%%%%%%%%%
 % Preliminares do documento            %
 %%%%%%%%%%%%%%%%%%%%%%%%%%%%%%%%%%%%%%%%

 \begin{titlepage}
  
  \hspace*{128pt}
  \textcolor{udcpink}{{\fontencoding{T1}\fontfamily{phv}\selectfont Facultade de Informática}}\\[-32pt]

  \begin{center}
    \includegraphics[scale=0.3]{imaxes/udc}\\[35pt]

    {\large PROGRAMACIÓN DE SISTEMAS \\
            Primeiro Cuadrimestre 2021/22
           } \\[50pt]
    
    \begin{huge}
      \begin{spacing}{1.3}
        \bfseries \titulo \\ \includegraphics[scale=0.6]{imaxes/icon.png}\\[35pt]
      \end{spacing}
    \end{huge}
  \end{center}
  
  \vfill
  
  \begin{flushright}
    {\large
    \begin{tabular}{ll}
      {\bf Estudante:} & \nomeA \\
      {\bf Estudante:} & \nomeB \\
      {\bf Estudante:} & \nomeC \\ % COPIA E PEGA ESTA LIÑA MÁIS VECES SE O PRECISAS
    \end{tabular}}
  \end{flushright}
  \rightline{A Coruña, \datasimple\today. Versión 0.2}
\end{titlepage}

 \pagenumbering{roman}
 \setcounter{page}{1}
 \bstctlcite{IEEEexample:BSTcontrol}

\begin{table}[h!]
\centering
\begin{tabular}{||c | c||} 
 \hline
 \textbf{Versión} & \textbf{Data}\\ [1ex] 
 \hline\hline
 0.1 & 04/10/2021\\
  [1ex] 
 \hline
 0.2 & 08/11/2021\\
  [1ex] 
 \hline
 0.3 & 13/01/2021\\
  [1ex] 
 \hline
  0.4 & 15/01/2022\\
  [1ex] 
 \hline
 0.5 & 18/01/2022\\
  [1ex] 
 \hline
 0.6 & 19/01/2022\\
  [1ex] 
 \hline
 \hline
 1.0 & 20/01/2022\\
  [1ex]
\end{tabular}
\caption{Táboa coas versións do proxecto.}
\label{table:1}
\end{table}

\let\cleardoublepage=\clearpage 

 \tableofcontents
% \listoffigures
% \listoftables

\vspace{30pt}


 %\cleardoublepage
 
 

 \pagenumbering{arabic}
 \setcounter{page}{1}
 
  \chapter{Introdución}
\label{chap:introducion}
\section{Obxetivos}
Este proxecto consiste en elaborar un xogo para o sistema operativo \textbf{ANDROID} basado no tradicional xogo do \textbf{Forcado}. O obxetivo da aplicación é poder xogar de maneira \textbf{individual} ou \textbf{multixogador} tratando de adiviñar a palabra secreta antes de que se forme por completo o forcado.\\
A aplicación estará enfocada ao xogo en local, é dicir, desde un único dispositivo (aínda que coa posibilidade de varias persoas involucradas). Aínda que contaremos cunha versión multixogador online ao final do proxecto.
%Logo, estudiarase a opción de poder xogar online a través dos distintos terminais interconectados entre sí.
\section{Motivación}
Visto o auxe do xénero do videoxogo na vida cotiá e a utilización de dispositivos móbiles estendido cada vez máis en distintos rangos de idades dos usuarios decidimos decantarnos por implementar un videoxogo para dispositivo móbil apto para todo tipo de público, tamén en auxe nos últimos anos e con tendencia alcista.


\section{Traballo relacionado}
Hai diversas aplicacións na Play Store nas que desenvolveron o mesmo xogo.
As dúas con maior impacto a nivel de descargas (> 10 millóns). Ambas aplicacións amosan unha interface familiar e intuitiva, podemos tomar como comparativa estas dúas aplicacións para partir do deseño, xa que este videoxogo en concreto non ofrecen moitas posibilidades neste aspecto. Nas que nos estamos baseando, son as seguintes:\\[100 pt]
\href{https://play.google.com/store/apps/details?id=com.tellmewow.senior.hangman&hl=es&gl=US}{Ahorcado desarrollado por Senior Games:}\\
\includegraphics[scale=0.65]{imaxes/app1.png}\\[12 pt]

\href{https://play.google.com/store/apps/details?id=com.quarzo.hangmanwords&hl=es&gl=US}{Ahorcado desarrollado por Quarzo Apps:}\\
\includegraphics[scale=0.65]{imaxes/app2.png}\\[12 pt]
 \let\cleardoublepage=\clearpage 


 \chapter{Análisis de requisitos}
\label{chap:requisitos}
\section{Funcionalidades}
Esta aplicación dispoñerá de dúas funcionalidades principais:
\begin{itemize}
    \item \textbf{Modo \textit{singleplayer}}: Esta funcionalidade consistirá en permitir ao usuario poder xogar de maneira individual. Neste caso, a aplicación escollerá unha palabra aleatoria dende un dicionario (podendo ser este ampliable polos usuarios) situado nos propios arquivos da aplicación e presentaralla ao usuario por pantalla para que trate de adiviñala.
    \item \textbf{Modo \textit{multiplayer}}: Nesta funcionalidade permitirase xogar a dous ou máis usuarios enfrentándose todos contra todos agrupados en diferentes equipos (que poden ser individuais). O funcionamento segue o mesmo modo que o de \textit{singleplayer} pero cunha ventana na que haxa que tocar a pantalla e que agoche o progreso de cada equipo entre cada turno para evitar trampas. Neste modo de xogo tamén se ofrecerá unha interface na que se declaren os equipos co seu nome (non se levará a cabo a función de rexistro de xogadores xa que non se considera relevante).
\end{itemize}

A prioridade de implementación das funcionalidades seguirá a orde descrita anteriormente, sendo o modo \textit{singleplayer} a primeira funcionalidade en ser implementada, o modo (\textit{multiplayer}) o segundo. \\
En traballos futuros pódense engadir novas funcionalidades como levar a conta de partidas gañadas e perdidas (pódense calcular estatísticas como porcentaxes de éxito ou número de intentos medios para obter unha victoria), tanto no modo \textit{singleplayer} como no \textit{multiplayer}. No primeiro modo, servirá a modo de histórico mentras que no segundo caso, definirá quen é o gañador ao final dun número de rondas.
 \let\cleardoublepage=\clearpage 
 \chapter{Planificación Inicial}
\label{chap:plan_inicial}

\section{Iteracións}
O proxecto fragmentarase na iteración de presentación do proxecto (\textit{brainstorming}, preparación dos documentos necesarios) dúas iteracións para o \textit{singleplayer} (desarrollo \textit{software} e probas) e outras dúas iguais para o \textit{multiplayer}.

Para cada modo de xogo, o primeiro que se fará é a vista principal do xogo e os fragmentos das diferentes modalidades que ofrece, como pode ser a elección do modo, ou obter o historial de partidas. Despois
comezarase coa creación da vista do xogo en si e os seus elementos e finalmente comprobarase con tests o seu correcto funcionamento, probar cando unha letra non pertence á palabra
a adiviñar, a condición de gañar ou perder a partida, etc.\\
Ao rematar a devandita iteración e antes da data de entrega do proxecto comezarase a  facer unha última iteración para implementar unha funcionalidade multixogador \textit{online} para partidas 1 contra 1 e as súas probas correspondentes. \\ \\
Como traballos futuros contémplase a elaboración dunha última iteración que implemente salas con máis de dous xogadores onde un deles faria como anfitrión ou servidor.
%\begin{enumerate}
%    \item \textbf{Modo singleplayer}
%    \item \textbf{Modo multiplayer}
%\end{enumerate}

\section{Responsabilidades}
Por dispoñibilidade temporal, a idea é que todos os integrantes do grupo implementen as funcionalidades xuntos xa que é improbable que se solapen os traballos de dous recursos diferentes. Todos os membros do grupo terán a mesma porcentaxe de responsabilidade aínda que o esperable é que os traballos se vaian asignando dinamicamente e sobre a marcha do proxecto segundo vaian xurdindo., axudarémonos do control de versións \textit{git} para facilitar o traballo.

\section{Fitos e entregables}
Haberá un total de 2
fitos, un para cada unha das funcionalidades a realizar cada unha cun mes de duración. Primeiro farase o modo \textit{singleplayer} e todo o que conleva, facer todas as vistas e fragmentos para que se poida xogar, se hai problemas ou non será a tarefa dos tests. No segundo , actualizarase a vista para que apareza o outro modo de xogo, e a vista será modificada respecto á do \textit{singleplayer} está pensado so para un xogador.%No terceiro fito, haberá que implementar o código que permita xogar ao forcado en dous diferentes dispositivos.


\section{Incidencias e plans de continxencia}
O principal risco que se vai ter en conta en conta debido á súa probable materialización son os contratempos á hora do desenvolvemento \textit{software}. Este risco está presente en case todos os proxectos \textit{software} e aínda máis cando se trata dun proxecto non profesional e con falta de experiencia na metodoloxía.\\
A solución aplicada para isto é tratar de adiantar o proxecto máximo posible coa mentalidade pesimista de que chegará algún contratempo que retrase o proceso.
%offline e se todo vai ben o online.
%No caso de non xurdiren problemas para implementar o modo online, intentarase ofrecer máis modalidades ou ítems de axuda. %nas funcionalidades singleplayer ou offline, 
%como poden ser moedas virtuais que compren pistas ou descubran letras para ter máis probabilidade de adiviñar a palabra.

 \let\cleardoublepage=\clearpage 
  \let\cleardoublepage=\clearpage 
 \chapter{Deseño}
\label{chap:deseño}
\section {Arquitectura proposta}
A arquitectura proposta aísla o código de desenvolvemento da base de datos e todas as funcións relacionadas con \textit{Room} do dicionario do resto da aplicación (modelo).\\
Tamén proponse unha interface para implementar os métodos en común entre os dous modos de xogo.


\section {Persistencia}
Almacenarase un dicionario coas palabras introducidas nunha base de datos. Non será posible abandoar o dicionario deixándoo baleiro e se se destrúe a aplicación restablécese. \\
\\
Coa axuda de \textit{Firebase} almacenaranse os usuarios rexistrados na aplicación co seu correo e a súa identificación. \\
\\
Plantéxase a idea de gardar tamén un historial das partidas para poder elaborar un cadro estatístico de cada perfil e poder ser consultado, pero esta idea quedará para traballos futuros.

\section {Vista}
Valóranse facer catro funcionalidades principais distintas coas súas correspondentes vistas. A primeira delas para o menú principal, constará de tres botóns. Un para o modo de un xogador, outra para o modo multixogador e outro para o acceso ao dicionario da aplicación. Amosaranse tamén o logo da aplicación e o nome.\\
\\
Seguindo a orde dos anteriores botóns, será necesaria unha vista para a partida de un solo xogador que constará dun debuxo do forcado con diferentes partes que irán modificando a súa visibilidade con cada fallo dinamicamente, un recadro para os ocos das letras da palabra a adiviñar e un botón por cada letra deseñado programaticamente no que ao pulsar unha letra fanse as correspondentes comprobacións e desactívase o botón pulsado. Habilitaríase o botón de atrás na \textit{ActionBar}. \\
\\
Para o multixogador serían necesarias máis vistas. Unha para o xogo en si, similar á anterior pero con algún detalle como un contador entre pulsacións e outra con dous botóns na que se poida crear unha sala para a partida ou unirse a unha mediante un código; se o usuario está autenticado amósase tamén un botón de pechar sesión. E para a parte de autenticación (soamente funcional dentro do multixogador) serían necesarias tres vistas principais. A primeira delas unha pantalla na que se amosan tres botóns para inicar sesión, rexistrarse e recuperar o contrasinal ademais de dous \textit{EditText} para o correo e o contrasinal. Se se preme no botón de recuperar o contrasinal só quedaría o \textit{EditText} do correo e un novo botón de acción. Por outro lado ao premer o botón de rexistrarse quedaría un botón de acción cos \textit{EditText} pertinentes para o rexistro dos datos. \\
\\
Por último, para o dicionario utilizarase un \textit{RecyclerView} coas palabras, que ao ser pulsadas poderán ser eliminadas, e un menú de opcións coa posibilidade de eliminar todas as palabras, restablecer o dicionario ou engadir unha palabra.\\
\\
Cabe destacar que en case tódalas actividades axudarémonos dalgún \textit{Toast} e bastantes \textit{AlertDialog} para mellorar a experiencia do usuario.

\section {Comunicacións}
As comunicacións implementadas no modo de un xogador baséanse en consultas a unha base de datos. \\
\\
Por outro lado, no modo multixogador precisaranse comunicacións entre dous dispositivos distintos a través de internet. Este método baséase no típico modelo no que un xogador crea unha sala virtual cun código e o outro xogadore únese introducindo ese código. Neste aspecto axudarémonos de \textit{Firebase} para elaborar unha autenticación dos usuarios e as comunicacións entre eles durante a partida. 
\\ \section {Sensores}
Para a utilización desta aplicación non é necesario ningún hardware específico do dispositivo utilizado exceptuando os sensores WiFi para obter conexión a internet no modo multixogador (que a aplicación seguiría sendo funcional co modo dun xogador sen o funcionamento do sensor).\\
Necesitaranse os permisos adecuados no manifesto da aplicación, non se necesitarán máis permisos para respectar a privacidade do usuario e a experiencia (non resulta cómodo ter que estar concedendo permisos a unha aplicación de terceiros). 

\section {Traballo en background} 
No modo multixogador plantéxase facer un thread  para o contador das partidas multixogador entre as seleccións de cada letra. \\
\\
Tódalas consultas á base de datos á base de datos local do dicionario como a \textit{Realtime Database} de \textit{Firebase} faranse mediante \textit{AsyncTask}s.

\let\cleardoublepage=\clearpage
  %\let\cleardoublepage=\clearpage 

 %%%%%%%%%%%%%%%%%%%%%%%%%%%%%%%%%%%%%%%%
 % Capítulos                            %
 %%%%%%%%%%%%%%%%%%%%%%%%%%%%%%%%%%%%%%%%


 %%%%%%%%%%%%%%%%%%%%%%%%%%%%%%%%%%%%%%%%
 % Apéndices, glosarios e bibliografía  %
 %%%%%%%%%%%%%%%%%%%%%%%%%%%%%%%%%%%%%%%%
 

  %\let\cleardoublepage=\clearpage
 
 \pagenumbering{arabic}
 \bibliographystyle{IEEEtranN}
 \bibliography{\bibconfig,bibliografia/bibliografia}
 
 

 
\end{document}

%%%%%%%%%%%%%%%%%%%%%%%%%%%%%%%%%%%%%%%%%%%%%%%%%%%%%%%%%%%%%%%%%%%%%%%%%%%%%%%%
