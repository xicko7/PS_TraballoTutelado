\chapter{Análisis de requisitos}
\label{chap:requisitos}
\section{Funcionalidades}
Esta apliación conterá principalmente tres funcionalidades principais:
\begin{itemize}
    \item \textbf{Modo \textit{singleplayer}}: Esta funcionalidade consistirá en permitir ao usuario poder xogar de maneira individual contra a propia CPU. Neste caso, a aplicación escollerá unha palabra aleatoria dende un diccionario (podendo ser este ampliable polos usuarios) situado nos propios arquivos da aplicación e presentaralla ao usuario por pantalla para que trate de adiviñala.
    \item \textbf{Modo \textit{multiplayer}}: Nesta funcionalidade permitirase xogar a dous ou máis usuarios enfrentándose todos contra todos agrupados en diferentes equipos (que poden ser individuais). O funcionamento segue o mesmo modo que o de \textit{singleplayer} pero cunha ventana na que haxa que tocar a pantalla e que agoche o progreso de cada equipo entre cada turno para evitar trampas. Neste modo de xogo tamén se ofrecerá unha interface na que se declaren os equipos co seu nome (non se levará a cabo a función de rexistro de xogadores xa que non se considera relevante).
\end{itemize}

A prioridade de implementación das funcionalidades seguirá o orde anterior descrito, sendo o modo \textit{singleplayer} a primeira funcionalidade a ser implementada, o modo (\textit{multiplayer}) o segundo. \\
Hai que ter en conta que dentro de cada funcionalidade incluirase outras que non son tan principais e que irán xurdindo a medida que se desenvolva a aplicación. Por exemplo, unha funcionalidade que se implementará será de levar a conta de partidas gañadas e perdidas, tanto no modo \textit{singleplayer} como no \textit{multiplayer}. No primeiro modo, servirá a modo de histórico mentras que no segundo caso, definirá quen é o gañador ao final de unhas rondas concretadas previamente polo usuario.
 \let\cleardoublepage=\clearpage 