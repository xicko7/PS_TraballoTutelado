\chapter{Planificación Inicial}
\label{chap:plan_inicial}

\section{Iteraciones}
Vaise implementar as funcionalidades según a complexidades de estas:
\begin{enumerate}
    \item \textbf{Modo singleplayer}
    \item \textbf{Modo multiplayer}
\end{enumerate}

\section{Responsabilidades}
A idea é que todos os integrantes do grupo implementen as funcionalidades xuntos. Todos os membros do grupo terán a mesma porcentaxe de responsabilidade.

\section{Hitos e entregables}
Haberá un total de 2 %3
hitos, un para cada unha das funcionalidades para realizar. A data para cada un dos hitos será de 1 mes e se probará o correcto funcionamento de cada unha.\\

No primeiro hito farase o modo \textit{singleplayer} e todo o que conleva, facer todas as vistas e fragmentos para que se poida xogar, se hai problemas ou non será a tarea do test.\\

No segundo hito, se actualizará a vista para que aparezca que hai outro modo de xogo ao que poder xogar, e a vista dentro da pantalla de xogo será modificada, xa que a do \textit{singleplayer} está pensado so para un xogador.\\

%No terceiro hito, haberá que implementar o código que permita xogar ao forcado en dous diferentes dispositivos. 
O tempo sobrante até a finalización do proxecto consistirá na refinación e correción dos últimos detalles da aplicación.


\section{Incidencias e plans de continxencia}
O primeiro que se fará é a vista principal do xogo e os fragmentos das diferentes modalidades que ofrece, como pode ser a elección do modo de xogo, ou obter o historial de partidas xogadas.
Despois comezarase coa creación da vista da pantalla de xogo con todos os seus  elementos e finalmente comprobarase con tests o seu correcto funcionamento, probar cando unha letra non pertence á palabra a adiviñar, a condición de gañar ou perder a partida, etc.\\
Todo o mencionado anteriormente, en modo \textit{singleplayer}, cando se comprobe o seu correcto funcionamento comezarase a implementar o modo \textit{multiplayer} .%offline e se todo vai ben o online.
\\
No caso de non xurdiren problemas para implementar o modo online, intentarase ofrecer máis modalidades ou ítems de axuda. %nas funcionalidades singleplayer ou offline, 
como poden ser moedas virtuais que compren pistas ou descubran letras para ter máis probabilidade de adiviñar a palabra.



